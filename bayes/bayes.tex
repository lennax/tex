\documentclass[12pt,a4paper]{article}
\usepackage[latin1]{inputenc}
\usepackage{amsmath}
\usepackage{amsfonts}
\usepackage{amssymb}
\author{Lenna X. Peterson}
\title{Bayesian Probabilities}
\begin{document}

\maketitle

\begin{equation}
P(E) = P(H,E) + P(\neg H, E)
\end{equation}

The probability of the evidence is equal to the probability of a true positive and the probability of a false positive.

The chance of a positive test result is equal to the probability of a true positive and the probability of a false positive.

\begin{equation}
P(E|H) = \frac{P(H,E)}{P(H)}
\end{equation}

The sensitivity is equal to the probability of a true positive divided by the probability of the hypothesis.

The probability of a positive test result given cancer is equal to the probability of a true positive divided by the probability of cancer.

\begin{align}
P(H|E) &= \frac{P(H,E)}{P(E)} \\
&= \frac{P(E|H) \times P(H)}{P(E)}
\end{align}

The probability of the hypothesis given the evidence is equal to the probability of a true positive divided by the probability of the evidence. 

The probability of the hypothesis given the evidence is equal to the sensitivity times the probability of the hypothesis divided by the probability of the evidence.

The probability of having cancer given a positive test result is equal to the probability of a true positive divided by the probability of a positive test result.

The probability of having cancer given a positive test result is equal to the sensitivity times the probability of cancer divided by the probability of a positive test result.

\end{document}